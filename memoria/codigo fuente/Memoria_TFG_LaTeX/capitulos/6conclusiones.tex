A modo de conclusión, volver a destacar la capacidad que tiene la aplicación de generar economía de kilómetro cero. Pero, no solamente haciendo publicidad de los establecimientos o actividades locales, sino que además, se podría integrar un sistema de comentarios o de valoraciones de los establecimientos publicitados, de los lugares naturales o incluso de actividades deportivas como puede ser experiencias con parapentes, etc. Esto generaría una comunidad en la que los usuarios podrían compartir sus experiencias, retar a sus amigos a participar en dichas actividades o visitar esos lugares. Y, al mismo tiempo, generaría retroalimentación para las empresas.

Gracias a la estructura del código, la aplicación podría adaptarse para funcionar con otras Islas o especializarse en ciudades importantes como La Laguna o Santa Cruz de Tenerife, incluyendo información sobre la historia de la ciudad, las plazas, las fuentes, etc.

Para lograr un acabado más profesional, se debería, a la hora de registrarse con cuenta de correo electrónico y contraseña, verificar el correo para evitar denegaciones de servicio. Y además, se podría añadir sistema de verificación en dos pasos y algún método de recuperación de contraseña.

Para añadirle mayor factor de gamificación podríamos establecer ``logros`` por:
\begin{itemize}
\item Ser el primer jugador en visitar un lugar.
\item Visitar cierta cantidad de veces un mismo sitio.
\item Visitar cierta cantidad de veces todos los sitios de la zona o de la Isla.
\item Acumular cierta cantidad de visitas en total de todos los sitios.
\end{itemize}

Otro posible añadido sería que los usuarios podrían tener la opción de enviar fotos y descripciones de lugares que no se encuentren dentro de la aplicación para, tras su revisión, ser añadidos al catálogo de la aplicación.

Como conclusión se puede extraer que durante esta memoria se ha demostrado la viabilidad económica, los múltiples beneficios y el potencial de éxito que posee este proyecto.